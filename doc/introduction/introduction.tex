\chapter{Introduction}

Unmanned Aerial Vehicles (UAV) are today widely used in ground observation, and by equipping them with different sensors they can be used in different situations. While the use of UAVs eases many cases of ground observation, there are some difficulties related to the attitude of the aircraft. When the sensor is attached directly to the aircraft the sensor will be coupled with the UAVs states, so what is captured by the camera depends on the angles of the UAV. Figure \ref{fig:intro_fig} illustrates how what is captured by the camera changes with the roll angle $\phi$.

\begin{figure}
	\import{/}{intro_fig.tex}
	\caption{An illustration of the issues related to UAV operation with fixed sensors.}
	\label{fig:intro_fig}
\end{figure}

A common solution to decouple the sensor from the UAV states is to attach the sensor to a gimbal which will counteract the movements of the UAV. While this is a good solution for decoupling, it raises some new issues regarding its weight and size. As one of the benefits of UAVs is their small size, the gimbal can quickly be too big and heavy for the UAV, and it may make the aerodynamics of the aircraft less effective. This usually leads to increased fuel consumption.

This paper will investigate methods that will ensure precise ground observation when a camera attached directly to the aircraft is used to observe both curved and piecewise linear paths on ground level, while also avoiding the extra costs associated with a gimbal. This will be accomplished by finding an optimal path that minimizes the deviance between what is to be observed and what is captured by the camera. The optimal path will be calculated by an offline intervalwise Nonlinear Model Predictive Control (NMPC) algorithm before the flight commences \cite{rhcKWON}. The control method is developed with the usage of a hyperspectral pushbroom camera that is fixed to the UAV in mind.

\import{./}{related.tex}
\import{./}{hyperspectral.tex}
\import{./}{contributions.tex}
\import{./}{thesis_outline.tex}