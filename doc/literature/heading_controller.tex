\subsection{Course controller}
The course of the aircraft is normally controlled by using the ailerons to roll the aircraft, with the resulting difference between the lift vectors of each wing causing the aircraft to turn. This strategy is the most common one used in larger, manned aircrafts as it causes little drag and it is comfortable for the passengers \cite{skidToTurnMills}. Since this strategy leads to problems when performing ground observations, alternative control strategies that reduces roll during course change exists.

\subsubsection{Rudder as a course control surface}
While the rudder is most commonly used to reduce the sideslip during flight, it can also be used to create sideslip which causes the aircraft to turn. The control method is a fairly common method to avoid roll during course change, and common for these controllers is that they use the ailerons too keep the wings level during flight.

%% RATC, FISHER %%
A controller using this control strategy was developed by Thomas Fisher in his paper "Rudder Augmented Trajectory Correction for Unmanned Aerial Vehicles to Decrease Lateral Image Errors of Fixed Camera Payloads" \cite{ratcFISHER}. In the paper the term 'Rudder Augmented Trajectory Correction' (RATC) is used for a controller using the rudder to change the course, and 'Aileron Only Trajectory Correction' (AOTC) for controllers using the ailerons as the course control surface. The implemented controller was a PD-controller simulated on a model of the Aerosonde UAV, and results focuses on image error when using a fixed camera.

%%When simulating the image error is modeled with two terms. The first term is the lateral image error which comes from the aircraft having a lateral error in its flightline so that it is not positioned directly over the intended path, which leads to image error. The second term is the error that comes from banking the aircraft. It is modeled using simple trigonometry, and it is worth noting that this error increases as the altitude above ground is increased.

The simulations showed that the course error of the two controllers were matching, both with and without wind. And unsurprisingly, the results show that the AOTC controller had much more variations in roll while the RATC controller had much more variations in sideslip. Simulations also showed that the RATC controller used much more input to its control surfaces, up to $400 \%$ more than the AOTC controller.

The difference in image error between the two controllers was bigger than the difference in course error. The image error was measured as the distance from the camera center point on the ground to the desired ground path, and while image errors for the RATC controller stayed at about $20$ m the AOTC controller had a RMS error over $300$ m. Field tests gave the same results and prove that RATC is a good choice for reducing image errors.

%It is worth noting from this paper that successive loop closure is not needed to implement the RATC. This is because the control design only has a singel transfer function between desired heading to control surface deflection. Since AOTC requires successive loop closure the AOTC controller will have a slower response compared to RATC.

%% Heading Control of a Fixed Wing UAV Using Alternate Control Surfaces %%
A similar approach was taken by Ahsan, Rafique and Abbas \cite{alternateSurfaceAhsan}, but a PID controller was used instead of a PD. The controller was created from a nonlinear model which had been linearized about a stable trim point, and the resulting rudder controller was compared with a controller using the aileron for heading control.

The simulations in this paper also show that when using the rudder as a control surface compared to aileron there is less overshoot and a lower steady state error. Bode plots of the two controllers show that the rudder based course controller has a gain margin of $-24.5 dB$ and a phase margin of $87.1 \degree$, while the aileron based controller has a gain margin of $-25.7 dB$ and a phase margin of $94 \degree$. This means that the two controllers have similar stability features.

%% Skid-to-Turn, MILLS %%
Mills, Ford and Mejias refers to a rudder-based course controller as a 'skid-to-turn' controller \cite{skidToTurnMills}. They use an UAV to survey a linear infrastructure, and the desired course is calculated based on the images of the infrastructure. The control method is called Image-Based Visual Servoing (IBVS), and it creates a model of the identified structure as a straight line that the UAV will follow while ensuring that the infrastructure stays within the cameras field of view (FOV). The controller was implemented as a PID-controller.

The controller was simulated and compared to a controller that banks the UAV to turn, and the results matches the previous results: the bank-to-turn controller reduces the track error the fastest, but the skid-to-turn controller causes less error in the images. The controllers were also tested in wind with similar results. One thing worth noting is that when the skid-to-turn controller were to intercept the structure with tailwind it resulted in a significant overshoot. In the image plane however, the error was still smaller than for the bank-to-turn controller.


\subsubsection{Course controller with constraints}

Banking the aircraft may cause the camera's FOV to move away from the point of interest, but when controlled the aircraft can be banked without losing the point of interest. In the paper "Low Altitude Road Following Constraints Using Strap-Down EO Cameras on Miniature Air Vehicle" \cite{constraintsEGBERT} constraints put on the UAV's roll and above ground level (AGL) ensure that the point of interest still stays within the camera's FOV when banking. In the paper the UAV tracks a roadway in a similar manner as was done in \cite{skidToTurnMills}, and the constraints are calculated with regards to the camera's horizontal field of view, the assumed road width and the expected turn angle of the road.

%The architecture of this system includes a 'Constraints Governor' that receives input from an image processing unit about the road it is following and telemetry input from the UAV about its position and heading. Based on these inputs the constraints ared calculated so that the road will stay within the camera's FOV. These constraints was forwarded to a previously made controller.

The system was tested by simulating how the UAV would follow two $90\degree$ turns without wind. The results show that the road was lost from the camera's FOV two times during the two turns. This happened because the system did not estimate the road's path well enough, and the paper argues that by pointing the camera forward the estimation can be improved.