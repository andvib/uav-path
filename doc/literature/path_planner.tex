\subsection{Path Planner}

In order to successfully and efficiently execute missions using UAVs, planning is essential. Path planners are used for this is reason, and there exists many different path planners depending on the situation. A path planner may, for example, be used to avoid controlled flight into surface in mountainous areas based on maps, or it can be used in search and rescue missions to calculate in which area it is most likely to find people. In this case the path planner will be used to generate a path that will ensure that the points of interest are not lost from the camera's FOV during flight.


\subsubsection{Path Planner for Ground Observation}
\label{ch:path_literature}
Many path planners today are based on the result Dubin presented in 1957 \cite{DUBIN}: the shortest path between two points in a two dimensional space consists of two circular arcs connected by a straight line. It has also been shown that the same principle can be used in three dimensions \cite{dubinsBEARD}. 

Dubin's path for UAVs can be used in several situations, and Lugo-Cardenas, Flores, Salazar, and Lozano \cite{dubinFixedWingLUGO} has written a paper on how a Dubin's path generator can be used to search for a missing person within a given area. The path is generated by a path generator, which then transmits the path to the path-following strategy that controls the low-level autopilot. The autopilot is responsible for maintaining a constant altitude and constant airspeed, while the path generator includes a constraint to ensure it does not generate a path which requires the UAV to exceed its maximum turning rate. When the UAV finds a point of interest, a path that circulates the point is generated. The path generator is simulated, and shows that Dubin's path is a valid choice for UAV operation.

As mentioned the aircraft course can be controlled by using the rudder. This strategy for airplane control can also be used in path planning, as was done by Yokoyama and Ochi \cite{skidPathYOKO}. A path planner based on Dubin's path was created with skid-to-turn dynamics in mind and it was compared with a path created by an optimization algorithm. The results show that the Dubin's path algorithm always returned a feasible path that is quasi-optimal. The Dubin's path algorithm was fast, with a mean computational time of $61.9\mu s$, which the report concludes is fast enough for the algorithm to be used for online calculations.