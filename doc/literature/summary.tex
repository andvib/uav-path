\subsection{Summary of Literature Review}
The literature review shows that the problem reviewed in this paper has already been asdressed, and that there already exists control methods that seeks to minimize image error. The review also shows that hyperspectral pushbroom cameras are being used for ground observation today and that these systems are moving from being large and heavy, to being reduced to a size more fitting for UAVs.

One additional point made in \cite{ratcFISHER} that is worth noting is that the RATC controller will ease the flight plans for ground observing. When using AOTC controllers for ground observing, extra measures often have to be taken to ensure that the entire area of interest is covered by the camera. For a typical $90 \degree$ turn this could be to fly past the turn, make complete circle in the opposite direction of the turn, and then continue on the path after the $90 \degree$ bend. When using the RATC controller, the flight path length and time was reduced by about $80 \%$, and it was estimated that the reduction in length and time will give a $75\%$ reduction in energy usage. This means that even though the paper concluded the RATC used $400 \%$ more input than the AOTC, the RATC will save time and maybe energy for complicated paths with many turns.