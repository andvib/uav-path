\section{MPC Method}

The MPC strategy can be broken down into three tasks \cite{mpcCAMACHO}:

\begin{enumerate}
	\item Predict the future outputs of the process for the given prediction horizon using past inputs to the process and the past measured states of the process, and by using the future control signals.
	\item Optimize an objective function in order to determine the future control signals that follows a given reference trajectory as closely as possible.
	\item Apply the optimal control signals to the process, and measure the resulting output so that it may be used to calculate the next prediction horizon in the first task.
\end{enumerate}

In short MPC problems are made up of three elements: Prediction model, objective function and constraints. The prediction model represents the model of the process that is to be controlled, and will in this case consist of the differential equations for the states of the UAV. The objective function is the function that is to be minimized by the optimization algorithm. In this case the function will be the distance from the camera centre point to the desired ground path, combined with some of the UAV states, that together will give a stable flight. The objective function represents the difference between the reference trajectory that the UAV states is to follow, and the current states of the UAV. The constraints are used to limit the values that either states or control inputs can take, and can prevent solutions that are not physically feasible.

A common mathematical formulation of the three elements that make up the optimization problem is shown in (\ref{eq:optimization_formulation}) \cite{nocedalOPTIMIZATION}. $f(x)$ represents the objective function that is subject to equality and inequality constraints respectively. In this thesis the differential equations describing the UAV model will be implemented as equality constraints, while the inequality constraints will be used to define the ranges the control inputs must be within.

\begin{equation}
	\label{eq:optimization_formulation}
	\begin{array}{rrclcl}
		\displaystyle \min_{x \in R^n} & \multicolumn{3}{l}{f(x)} \\
		\textrm{s.t}
		& c_i(x) = 0, i \in \epsilon, \\
		& c_i(x) \geq 0, i \in I.
	\end{array}
\end{equation}