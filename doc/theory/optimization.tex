\chapter{Model Predictive Control}
\label{ch:optimization}

Model Predictive Control (MPC) is a term used to describe control methods that uses knowledge about the process to calculate the future control inputs to the system in order to follow a reference trajectory \cite{mpcCAMACHO}. In this chapter the equations for a MPC "system" that seeks to minimize the distance between the camera centre point and the ground path that is to be observed. A linear state space-model for the UAV will be used to predict the future states and control inputs.


\section{MPC Method}

The MPC strategy can be broken down into three tasks \cite{mpcCAMACHO}:

\begin{enumerate}
	\item Predict the future outputs of the process for the given prediction horizon using past inputs to the process and the past measured states of the process, and by using the future control signals.
	\item Optimize an objective function in order to determine the future control signals that follows a given reference trajectory as closely as possible.
	\item Apply the optimal control signals to the process, and measure the resulting output so that it may be used to calculate the next prediction horizon in the first task.
\end{enumerate}

In short MPC problems are made up of three elements \cite{mpcCAMACHO}: Prediction model, objective function and the control law. The prediction model represents the model of the process that is to be controlled, and will in this case consist of the differential equations for the states of the UAV. The objective function is the function that is to be minimized by the optimization algorithm, in this case this will be the distance from the camera centre point to the desired ground path together with some of the UAV states that will give a stable flight. The objective function represents the reference trajectory that the UAV is to follow. The control law introduces constraints on the problem, reducing the number of feasible solutions. This constraints can be put on either the states or the control inputs for the UAV.

A mathematical formulation of the three elements that make up the optimization problem is shown in \ref{eq:optimization_formulation} \cite{nocedalOPTIMIZATION}. $f(x)$ represents the objective function that is subject to equality and inequality constraints respectively. The equality constraints are used to represent the UAV model, while the inequality constraints represent the constraints used for the control law. The control law may also consist of equality constraints.

\begin{equation}
	\label{eq:optimization_formulation}
	\begin{array}{rrclcl}
		\displaystyle \min_{x \in R^n} & \multicolumn{3}{l}{f(x)} \\
		\textrm{s.t}
		& c_i(x) = 0, i \in \epsilon, \\
		& c_i(x) \geq 0, i \in I.
	\end{array}
\end{equation}

The equations shown in equation \ref{eq:optimization_formulation} is the equations needed for any kind of optimization problems, also MPC problems. A MPC problem differs from a general optimization problem in the cost function, where the optimization horizon is expressed \cite{mpcCAMACHO}:

\begin{equation}
	\label{eq:mpc_cost}
	f(x) = J(N_1, N_2, N_3) = \sum_{j=N_1}^{N_2} \delta(j)[\hat{y}(t+j|t)-w(t+j)]^2 + 
	\sum_{j=1}^{N_u}\lambda(j)[\Delta u(t+j-1)]^2.
\end{equation}

The first term of equation \ref{eq:mpc_cost} represents the costs from the states of the model, and the second term represents the cost of the control effort. In the first term $\hat{y}$ is the value of the prediction model, which is compared to the desired trajectory $w$. In the second term the changes in control $\Delta u$ is expressed. The changes in control is used instead of the value of the control signal itself, since the steady state of the control signal may not be zero. $\delta$ and $\lambda$ are weighting variables which offers a way of tuning the MPC. The three different $N$ coefficients defines the horizon over which the states and the control effort should be optimized.


\section{Least-Squares Problem}

\begin{equation}
	\label{eq:lsq}
	f(x) = \frac{1}{2} \sum_{j=1}^m r_j^2(x)
\end{equation}

In many applications the objective functions is formulated as a least-square (LSQ) problem and the general form of an LSQ is shown in equation \ref{eq:lsq} \cite{nocedalOPTIMIZATION} where $r_j$ is referred to as the residual function, and it is the function that we seek to minimize. This method is often used when problem involves measurements that is expected to follow a model that is known in advance. By formulating this as a LSQ problem and minmizing the resulting optimization problem, it is possible to find parameters that minimizes the difference between the model and the measurements. 


\section{Problem Definition}

The control problem in this thesis will be solved by using an offline MPC to generate an optimal path that will reduce the image error when using a fixed camera to survey a ground track. An offline MPC means that the initial state of the MPC is not a measurement of the UAV states, but rather the result of a simulation of the UAV. The offline MPC is still similar to the online version as it uses the model of the UAV to predict the future states.

Rawlings \& Mayne \cite{mpcMAYNE} refers to this kind of problem as a \textit{deterministic problem} since there is no uncertainty in the system. Because there is no uncertainty in the problem feedback is not needed as it does not present any new information. They also state that an MPC action from a system like this is the same as the action from a \textit{receding horizon control law} (RHC), which is another kind of predictive control.

Altough the feedback is not needed to give new information, it eases the computational load of the control problem as optimizing the path over a long time horizon leads to a very complicated problem. THIS NEEDS SOURCE AND BETTER WRITING.

The linear decoupled 12 DOF UAV model presented by Beard \& McLain \cite{uavBEARD} will be implemented. This model is associated with the following states and control inputs:

\begin{subequations}
\begin{equation}
	\mathbf{x} =
	\begin{bmatrix}
		p_N \hspace{5pt} p_E \hspace{5pt} h \hspace{5pt}
		u \hspace{5pt} v \hspace{5pt} w \hspace{5pt}
		\phi \hspace{5pt} \theta \hspace{5pt} \psi \hspace{5pt}
		p \hspace{5pt} q \hspace{5pt} r
	\end{bmatrix}^T
\end{equation}
\begin{equation}
	\mathbf{u} =
	\begin{bmatrix}
		\delta_e \hspace{5pt} \delta_a \hspace{5pt} \delta_r \hspace{5pt} \delta_t
	\end{bmatrix}^T.
\end{equation}
\end{subequations}


\subsection{Prediction Model}

The prediction model relates to the equality constraints of equation \ref{eq:optimization_formulation} in the form of differential equations. Based on the control inputs and current states, $\mathbf{\dot{x}}$ is calculated by the differential equation. The attitude angles will be expressed in Euler angles. Even though quaternions offer more efficient computations and no gimbal lock \cite{uavBEARD}, this optimization will be run on an offboard computer/offline so that computation capacity is not a big issue and the UAV is not going to undergo any extreme maneuvers so that a gimbal lock should never occur.


\subsection{Objective Function}

The parametrized path that the camera is to observe will be given as a function $P(k)$ where $k$ is the path paramter, and P is a $2xn$ matrix where the elements represent the position in $p_E$ and $p_N$

\begin{equation}
	\mathbf{P}(k) = 
	\begin{bmatrix}
		\mathbf{p}_E(k) \\ \mathbf{p}_N(k)
	\end{bmatrix}
	=
	\begin{bmatrix}
		x_0 & x_1 & x_2 & \hdots & x_n \\
		y_0 & y_1 & y_2 & \hdots & y_n
	\end{bmatrix}.
\end{equation}

The residual function $r$ will be defined as the difference between the observation path and the camera centre point $\mathbf{c}^n$ from equation \ref{eq:camera_position_ned}:

\begin{equation}
	r_j(x) = ||\mathbf{P}(k) - \mathbf{c}^n(x)||.
\end{equation} 


\subsection{Control Law}

The only physical constraints in this optimization problem is constraints on the control inputs, as shown in equation \ref{eq:control_constraint}. This inequality constraint ensures that the control surfaces of the UAV simulated in the optimization do not excede what the UAV is physically capable of.

\begin{equation}
	\label{eq:control_constraint}
	\mathbf{u}^{low} \leq \mathbf{u} \leq \mathbf{u}^{high}
\end{equation}
% Should maybe come from a source? Different formulation?